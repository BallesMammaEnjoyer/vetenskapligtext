\documentclass[11p]{article}
% Packages
\usepackage{amsmath}
\usepackage{graphicx}
\usepackage[swedish]{babel}
\usepackage[
    backend=biber,
    style=authoryear-ibid,
    sorting=ynt
]{biblatex}
\usepackage[utf8]{inputenc}
\usepackage[T1]{fontenc}
%Källor
\addbibresource{mall.bib}
\graphicspath{ {./images/} }

\title{Text om tidiga vetenskapsmän}
\author{Mille Tedebring Thysell }
\date{\today}

\begin{document}

    \begin{titlepage}
        \begin{center}
            \vspace*{1cm}

            \Huge
            \textbf{Text om tidiga vetenskapsmän}

            \vspace{0.5cm}
            \LARGE


            \vspace{1.5cm}

            \textbf{Mille Tedebring Thysell}

            \vfill


            Fysik 1

            \vspace{0.8cm}

            \includegraphics[width=0.4\textwidth]{../images/NTI Gymnasiet_Symbol_print_svart.png}

            \Large
            Teknikprogrammet\\
            NTI Gymnasiet\\
            Umeå\\
            \today

        \end{center}
    \end{titlepage}
    

    Fysik var bara massor av gissningar under majoriteten av männsklig historia och var ofta kopplad med religion,
    det var kring under 1500-talet då det började faktiskt bli en vetenskap. Detta var tack vare många personer som
    tänkte, mätte och beräknade för att hitta hur det faktiskt var. Personerna som berättas om i detta PM är de som folk
    anser var de viktigaste bland de tidiga fysikerna.\\

    \large Nicolaus Copernicus 1473-1543\\
    \normalsize
    Nicolaus Copernicus var en polsk man som var ett "universalgeni", en person som är väldigt
    kunnig inom flera specialistområden. Han är mest känd för att ha skapat en model för solsystemet där solen
    är placerad i mitten med planeterna roterandes omkring. Innan hans teori så trodde många att jorden var i
    mitten med planeterna, solen och stjärnorna placerade omkring i ett fäste.Han skapade även
    kvantitetsteorin inom ekonomi.\footnote{https://www.britannica.com/biography/Nicolaus-Copernicus}  \\
    "[Nicolaus Copernicus] var en polsk astronom, matematiker, jurist, ekonom, militärstrateg, tolk, ambassadör, läkare,
    astrolog och kanik."\footnote{https://sv.wikipedia.org/wiki/Nicolaus_Copernicus}\\

    \large Tycho Brahe 1546-1601\\
    \normalsize Tycho Brahe, född i Danmark, var en astronom som är välkänd för hans utvecklan av mer noggranna
    vetenskapliga instrument och för hans dokumentation av stjärnor och deras plats i himlen. Han  \footnote{https://www.britannica.com/biography/Tycho-Brahe-Danish-astronomer}
    Han utvecklade astronomi som en vetenskap väldigt
    mycket med noggranna mätningar och data och bättre instrument. \footnote{https://en.wikipedia.org/wiki/Tycho_Brahe}
    \\


    \large Galileo Galilei 1563-1642\\
    \normalsize Galileo Galilei var född nära Florens i Italien.\footnote{https://sv.wikipedia.org/wiki/Galileo_Galilei} Han var en skicklig instrumentskapare och skapade
    nogranna och starka teleskop, de var starka nog att han kunde producera tidiga teckningar av månen som
    visade att den inte var ett perfekt klot och hade berg och dalar. Han var en av de första som kunde dokumentera
    månen med sån kvalitet och gjorde många fler observationer, till exempel så publicerade han om himlakroppar som
    kretsade kring Jupiter och solfläckar på solytan. När Galileo publicerade ett manuskript som höll med om den Koperniska
    synen på solsystemet så hamnade han i konflikt med kyrkan, de varnade han och Koperniska läror förbjöds. År 1632
    så publicerade han boken "Dialogo sopra i due massimi sistemi del mondo" (Dialog om de två världssystemen), den handlade om ifall Jorden eller Solen
    var i mitten av allt. Då blev Galileo dömd till husarrest för resten av hans liv, han dog i 1642.\footnote{https://fof.se/artikel/2009/5/det-han-sag-fick-inte-vara-ratt/}\\



    \large Johannes Kepler 1571-1630\\
    \normalsize Johannes Kepler var en  \\


    \large Isaac Newton \\
    \normalsize lorem \\


    \printbibliography

\end{document}
